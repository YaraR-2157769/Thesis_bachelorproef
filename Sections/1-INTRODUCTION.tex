\section{INTRODUCTION}
Gravitationele lensing is een effect dat optreedt wanneer licht afgebogen wordt door het gravitatieveld van massieve objecten, zoals sterrenstelsels of donkere materie \cite{informationesoorg-no-date}. De afbuiging van het licht kan zowel Newtoniaans als gravitationeel afgeleid worden. De Newtoniaanse afleiding voor een hyperbolische baan is te vinden in \cref{appendix: newton}. In tegenstelling tot klassieke lensing, waarbij één object wordt afgebeeld op één ander object, kan bij gravitationele lensing één object afgebeeld worden op meerdere objecten \cite{unknown-author-2022}. Dit wordt bepaald door de lensvergelijking. De afleiding van de lensvergelijking in 1D is te vinden in \cref{appendix:lensvergelijking}.\\ \\
Door de afbuiging van het licht als gevolg van het zwaartekrachtveld van een zware massa (denk aan een sterrenstelsel, donkere materie ...) 
worden de objecten niet waargenomen op de plaats waar ze effectief staan \cite{lea-2023}. Vanuit de plaats waarop het object waargenomen wordt, kan de effectieve positie bepaald worden. Stel nu dat het object afgebeeld wordt op meerdere deelobjecten, dan moeten alle deelobjecten teruggerekend worden. Dit kan op een zeer analoge manier als bij één object. \\ \\
Soms kan het voorkomen dat een object afgebeeld wordt achter de zware massa zelf. Dan is de kans dat het beeld niet gevonden wordt reëel. Het is echter wel noodzakelijk om alle beelden te hebben om de effectieve positie van het object te kunnen achterhalen. Het doel van deze bachelorproef is het wegwerken van de zware massa op de voorgrond, om het beeld dat op de achtergrond staat waar te kunnen nemen. Het vinden van het beeld dat op de achtergrond staat kan ook heel nuttig zijn om informatie te bekomen over de massaverdeling van de zware massa. Zeker als onze zware massa donkere materie is, kan het interessant zijn om meer te weten te komen over de massaverdeling ervan. \\ \\
Er bestaan al een heel aantal manieren waarop dit gedaan kan worden. Voorbeelden hiervan zijn IMFIT \cite{chen-2020}, \cite{erwin-2015} en GALFIT \cite{peng-2010}, \cite{unknown-author-no-date-galfit}.
In deze bachelorproef wordt er gebruik gemaakt van Bayesiaanse statistiek en deep learning.



