\clearpage
\appendix
\addcontentsline{toc}{section}{Appendices}
\section*{Appendices}
\section{Newtoniaanse bepaling afbuighoek als gevolg van een gravitationeel veld}
\label{appendix: newton}
Er wordt vertrokken vanuit de bewegingsvergelijkingen voor een deeltje in een potentiaal, gecreëerd door het zwaartekrachtveld. Er wordt aangetoond dat de oplossingen van de bewegingsvergelijking kegelsneden (cirkel, ellips, hyperbool, parabool) zijn. Deze afleiding is gebasserd op het handboek Analytical Mechanics van Hand en Finch, \cite{hand-1998}.
\subsection{De baan van een deeltje in een zwaartekrachtveld zijn kegelsneden}
Voor gravitatie wordt de potentiële energie gegeven door
$$V(r)=-\frac{GmM}{r}$$
De effectieve potentiële energie is gegeven door
$$V_{eff}(r)=-\frac{k}{r}+\frac{l^{2}}{2\mu r^{2}}$$
Hierin is $k=GmM$.\\
$\mu$ is de gereduceerde massa, deze wordt gevonden uit $\frac{1}{\mu}=\frac{1}{M}+\frac{1}{m}$\\
De bewegingsvergelijking wordt dan gegeven door
$$\mu\ddot{r}=\frac{l^{2}}{\mu r^{3}}-\frac{k}{r^{2}}$$
Deze vergelijking is niet lineair. Er wordt een nieuwe variabele ingevoerd. Stel $u=\frac{1}{r}$ en $\dot{\phi}=\frac{l}{\mu r^{2}}$, dit volgt uit behoud van draaimoment. Er volgt dan
$$\mu r^{2}\dot{\phi}=l;\ \ \ \ \ \mu r^{2}d\phi=ldt;\ \ \ \ \ dt=\frac{\mu}{lu^{2}}d\phi$$
Dan geldt er dat
\begin{align}
\dot{r}&=\frac{d}{dt}r=\frac{d}{dt}\left(\frac{1}{u}\right) \nonumber \\
&=-\frac{1}{u^{2}}\frac{du}{dt} \nonumber \\
&=-\frac{1}{u^{2}}\frac{du}{d\phi}\frac{d\phi}{dt} \nonumber \\
&=-\frac{1}{u^{2}}\frac{lu^{2}}{\mu}\frac{du}{d\phi} \nonumber \\
&=-\frac{l}{\mu}\frac{du}{d\phi} \nonumber 
\end{align}
De 2de afgeleide is gegeven door
\begin{align}
\ddot{r}&=\frac{d}{dt}\left[-\frac{l}{\mu}\frac{du}{dt}\right] \nonumber \\
&=-\frac{l}{\mu}\frac{d}{d\phi}\left(\frac{du}{d\phi}\right)\frac{d\phi}{dt} \nonumber \\
&=-\frac{l}{\mu}\frac{lu^{2}}{\mu}\frac{d^{2}u}{d\phi^{2}} \nonumber 
\end{align}
De bewegingsverdelijking wordt dan een differentiaalvergelijking, gegeven door
$$\frac{d^{2}u}{d\phi^{2}}+u=\frac{k\mu}{l^{2}}$$
De oplossing van deze differentiaalvergelijking is gegeven door
$$u(\phi)=\frac{\mu k}{l^{2}}+A\cos(\phi)$$
Er wordt overgegaan op Cartesische coördinaten. Stel $p=\frac{l^{2}}{\mu k}$ en stel $\epsilon=pA\geq0$. Dan geldt er
$$pu=1+\epsilon\cos(\phi)$$
$$p=r+r\epsilon \cos(\phi)$$
\begin{align}
    (p-\epsilon x)^{2}=x^{2}+y^{2} 
    \label{for:baan}
\end{align}
De energie is gegeven door
\begin{align}
    E&=\frac{1}{2}\mu(\dot{r}^{2}+r^{2}\dot{\phi}^{2})-\frac{k}{r}\nonumber \\
    &=\frac{\mu k^{2}}{2l^{2}}(\epsilon^{2}-1)
    \label{for: energie}
\end{align}
Merk op dat de energie afhankelijk is van $\epsilon$. Als $\epsilon > 1$ wordt de baan van een hyperbool gevonden. Dat wordt aangetoond vanuit \cref{for:baan}.

Als $\epsilon > 1$, dan volgt uit \cref{for: energie} dat de energie groter is dan 0. De baan is dan gegeven door
\begingroup\makeatletter\def\f@size{7}\
\def\maketag@@@#1{\hbox{\m@th\large\normalfont#1}}%
$$p^{2}+\epsilon^{2}x^{2}-2\epsilon px-x^{2}-y^{2}=0$$
$$\left(1-\epsilon^{2}\right)x^{2}+2\epsilon px+y^{2}-p^{2}=0$$
\begin{align}
&\left(1-\epsilon^{2}\right)\left(x^{2}+\frac{2\epsilon px}{(1-\epsilon^{2})}+\frac{\epsilon^{2}p^{2}}{(1-\epsilon^{2})^{2}} \frac{\epsilon^{2}p^{2}}{(1-\epsilon^{2})^{2}}\right)
    +y^{2}-p^{2}=0    \nonumber
\end{align}
$$\left(1-\epsilon^{2}\right)\left(x+\frac{\epsilon p}{(1-\epsilon^{2})}\right)^{2}+y^{2}-\frac{\epsilon^{2}p^{2}}{(1-\epsilon^{2})}-p^{2}=0$$
$$\left(1-\epsilon^{2}\right)\left(x+\frac{\epsilon p}{(1-\epsilon^{2})}\right)^{2}+y^{2}=\frac{p^{2}}{(1-\epsilon^{2})}$$ \endgroup
Dit kan ook nog geschreven worden als
$$\frac{(x-x_{c})^{2}}{a^{2}}-\frac{y^{2}}{b^{2}}=1$$
Met: \\
$x_{c}=\frac{-\epsilon p}{1-\epsilon^{2}}$ \\
$p=\frac{l^{2}}{\mu k}$\\
$\epsilon = pA$\\
$a=\frac{p}{1-\epsilon^{2}}$\\
$b=\frac{p}{\sqrt{1-\epsilon^{2}}}$\\
Dit komt overeen met de baan van een hyperbool. De andere kegelsneden worden buiten beschouwing gelaten.
\subsection{De bijhorende afbuighoek voor een hyperbolische baan}
De verandering in kinetische energie van het deeltje met massa $m$ en snelheid $v$ in de potentiaal (per eenheidsmassa) is gegeven door 
\begin{equation}
    E=\frac{1}{2}mv^{2}
    \label{for:energie}
\end{equation}
Het draaimoment per eenheidsmassa is gegeven door
\begin{equation}
    L=bv
    \label{for:draaimoment}
\end{equation}
Deze twee grootheden kunnen ook geschreven worden in termen van de excentriciteit. Vertrekkende van de energie gevonden in \cref{for: energie}, en gebruikmakende van $p=\frac{l^{2}}{\mu k}$ wordt er gevonden dat
$$E=\frac{k}{2p}(\epsilon^{2}-1)$$
a wordt gedefinieerd als 
$$a=\frac{p}{\epsilon^{2}-1}$$
Zo wordt de energie per eenheidsmassa gevonden.
\begin{align}
    E &= \frac{k}{2ma}\nonumber \\
    &= \frac{GM}{2a}
    \label{for:E}
\end{align}
Voor het draaimoment per eenheidsmassa geldt
\begin{align}
    L^{2} &= r^{2}v^{2}\nonumber \\
    &= r^{2}\cdot 2E \nonumber \\
    &= \frac{GM}{a}\cdot b^{2}
    \label{for:L2}
\end{align}
\begin{figure}
    \centering
    \includegraphics[width=0.95\linewidth]{Figures/hoek_hyperbool.png}
    \caption{De halve hoek van een hyperbolische baan. Figuur van Weber, \cite{weber-no-date}}
    \label{fig: halve hoek}
\end{figure}
De halve hoek van de hyperbolische baan is te zien in \cref{fig: halve hoek}. Daaruit volgt eenvoudig dat:
\begin{equation}
    tan(\alpha) = \frac{b}{a}
    \label{for: hoek}
\end{equation}
\cref{for: hoek} wordt herschrijven in termen van L (\cref{for:L2}) en E (\cref{for:E}).
Merk op dat
$$E\cdot L^{2}=\frac{G^{2}M^{2}b^{2}}{2a^{2}}$$
$$\frac{2E\cdot L^{2}}{G^{2}M^{2}}=\frac{b^{2}}{a^{2}}$$
$$\frac{\sqrt{2E}L}{GM}=\frac{b}{a}$$
Invullen van \cref{for:energie} en \cref{for:draaimoment} geeft
\begin{align}
    \frac{b}{a}&=\frac{\sqrt{2\cdot\frac{1}{2}v^{2}}rv}{GM}\nonumber \\
     &= \frac{rv^{2}}{GM}\nonumber\\
\end{align}
Voor de wordt gevonden dat
\begin{equation}
    \tan(\alpha) =\frac{rv^{2}}{GM} 
    \label{for:tangens}
\end{equation}
De afbuighoek is gegeven door de hoek tussen de asymptoten. Uit \cref{fig: halve hoek} volgt dan
$$\theta = \pi - 2\alpha$$
Omvormen naar $\alpha$ geeft
$$\alpha = \frac{\pi}{2}-\frac{\theta}{2}$$
De tangens nemen
\begin{equation}
    \tan(\alpha) = \tan\left(\frac{\pi}{2}-\frac{\theta}{2}\right)
    \label{for:tan}
\end{equation}
Het herschrijven van de tangens geeft
\begin{align}
    \tan\left(\frac{\pi}{2}-\frac{\theta}{2}\right)&=\frac{\sin(\frac{\pi}{2}-\frac{\theta}{2})}{\cos(\frac{\pi}{2}-\frac{\theta}{2})}\nonumber\\
    &= \frac{\cos(\frac{\theta}{2})}{\sin(\frac{\theta}{2})}\nonumber \\
    & = \cot(\frac{\theta}{2})
    \label{for:cot}
\end{align}
Invullen van \cref{for:cot} in \cref{for:tan} geeft
\begin{equation}
    tan(\alpha) = \frac{1}{tan(\frac{\theta}{2})}
    \label{for:afbuighoek}
\end{equation}
Invullen van \cref{for:tangens} en Taylor toepassen voor kleine $\theta$ geeft
$$\frac{1}{tan(\frac{\theta}{2})} = \frac{rv^{2}}{GM}$$
$$\frac{1}{\frac{\theta}{2}}=\frac{rv^{2}}{GM}$$
$$\frac{\theta}{2} = \frac{GM}{rv^{2}}$$
$$\theta = \frac{2GM}{rv^{2}}$$
Als de afbuiging van licht beschouwd wordt is de snelheid gegeven door de snelheid van het licht. Dan is de afbuighoek gegeven door \cref{for: afbuighoek}.
\begin{equation}
    \theta = \frac{2GM}{rc^{2}}
    \label{for: afbuighoek}
\end{equation}
Dit was nu de afleiding voor de hyperbool. Merk op dat dit de Newtoniaanse afleiding was. Als de relativistische correcties mee in rekening gebracht worden, wordt in de teller een extra factor 2 gevonden. Er wordt dan gevonden dat
$$\theta=\frac{4GM}{rc^{2}}$$
\section{De lensvergelijking}
\label{appendix:lensvergelijking}
De werking van de lens kan benaderd worden als een geometrisch probleem in twee dimensies. De voorstelling van de werking van de lens is te zien in \cref{fig:lensvgl}
\begin{figure}[H]
    \centering
    \includegraphics[scale=0.5]{Figures/Lensvergelijking.png}
    \caption{Schets van de werking van de lens in twee dimensies. Figuur van Jodrell Bank Centre for astrophysics, \cite{jodrell-bank-centre-for-astrophysics-no-date}.}
    \label{fig:lensvgl}
\end{figure}\mbox{}
Om de lensvergelijking af te leiden wordt er gezocht naar de afstanden $OQ$, $OQ'$ en $QQ'$. Er wordt steeds aangenomen dat de hoeken klein zijn. Uit de figuur volgt er dat
$$\tan(\beta) = \frac{|OQ|}{D_{s}}$$
$$|OQ| = D_{s}\tan(\beta)$$
Doordat de hoeken klein zijn kan er taylor rond $\beta=0$ toegepast worden. Er wordt dan gevonden dat
\begin{equation}
  |OQ| = D_{s}\beta  
  \label{for:OQ}
\end{equation}
Analoog voor $|OQ'|$ geldt er
\begin{equation}
   |OQ'| = D_{s}\theta 
   \label{for:OQ'}
\end{equation}
Om de afstand $|QQ'|$ te bepalen wordt er gewerkt in de driehoek $\Delta PQQ'$. Een close-up van de driehoek is te zien in \cref{fig:driehoek}.
\begin{figure}[h]
    \centering
    \includegraphics[scale=0.45]{Figures/driehoek.png}
    \caption{Close-up van de desbetreffende driehoek}
    \label{fig:driehoek}
\end{figure}
Er wordt opgemerkt dat $|QQ'| = |Qq|+|qQ'|$. Die twee afstanden zijn makkelijk te vinden uit de tangens. De hoek $\alpha$ wordt gesplitst in twee hoeken (die optellen tot $\alpha$). Zo wordt een rechthoekige driehoek verkregen. De eerste hoek is dan $\frac{\alpha}{x}$. De tweede hoek is gegeven door $\frac{(x-1)\alpha}{x}$. Zo zijn de twee afstanden gegeven door
$$\tan\left(\frac{\alpha}{x}\right)=\frac{|Qq|}{D_{ls}}$$
$$|Qq|=\tan\left(\frac{\alpha}{x}\right)D_{ls}$$
Analoog wordt er gevonden
$$\tan\left(\frac{(x-1)\alpha}{x}\right)=\frac{|qQ'|}{D_{ls}}$$
$$|qQ'|=\tan\left(\frac{(x-1)\alpha}{x}\right)D_{ls}$$
De som is dan gegeven door
$$|QQ'|=D_{ls}\left(\tan\left(\frac{\alpha}{x}\right)+\tan\left(\frac{(x-1)\alpha}{x}\right)\right)$$
Doordat de hoek $\alpha$ klein is kan er Taylor toegepast worden op de tangens, rond 0. Dan wordt er gevonden dat
\begin{align}
    |QQ'| &=D_{ls}\left(\frac{\alpha}{x}+\frac{x\alpha}{x}-\frac{\alpha}{x}\right) \nonumber\\
    &=D_{ls}\alpha
    \label{for:QQ'}
\end{align}
Nu de afstanden $|OQ|, |OQ'|$ en $|QQ'|$ gekend zijn (zie \cref{for:OQ}, \cref{for:OQ'} en \cref{for:QQ'}), kan de lensvergelijking gevonden worden. Het is eenvoudig in te zien dat $|OQ'|=|OQ|+|QQ'|$. Daaruit volgt
\begin{align}
    D_{s}\theta &=D_{s}\beta + D_{ls}\alpha \nonumber\\
    \theta &= \beta + \frac{D_{ls}}{D_{s}}\alpha \nonumber\\
    \beta &= \theta - \frac{D_{ls}}{D_{s}}\alpha
    \label{for: lensvgl}
\end{align}
\cref{for: lensvgl} wordt de lensvergelijking in 1D genoemd.
\onecolumn
\mbox{}
\section{Opbouw afschatten parameters van een sersic}
Voor de afschatting van de parameters van deze appendix werden 8\_500\_000 samples gegenereerd, waarvan er 1\_500\_000 weggegooid werden.
\label{appendix: opbouw}
\begin{figure}[H]
    \begin{minipage}{0.49\linewidth}
        \includegraphics[width=0.98\textwidth]{Figures/sersic_parameters_metropolis_8500000_1500000_50_y0 (1).png}
        \subcaption{2 onbekende parameters}
    \end{minipage}
    \begin{minipage}{0.49\linewidth}
        \centering
        \includegraphics[width=0.98\textwidth]{Figures/sersic_parameters_metropolis_8500000_1500000_50_ellips.png}
        \subcaption{3 onbekende parameters} 
    \end{minipage}
    \begin{minipage}{0.49\linewidth}
        \includegraphics[width=0.98\textwidth]{Figures/sersic_parameters_metropolis_8500000_1500000_50_amplitude (1).png}
        \subcaption{4 onbekende parameters}
    \end{minipage}
    \begin{minipage}{0.49\linewidth}
    \includegraphics[width=0.98\textwidth]{Figures/sersic_parameters_metropolis_8500000_1500000_50_theta (2).png}
    \subcaption{5 onbekende parameters}
    \end{minipage}
    \caption{Het proces waarin telkens meer onbekende toegevoegd worden. Er wordt begonnen met drie ongekende parameters, en toegewerkt naar vijf onbekenden.}
    \label{fig:4 onbekenden}
\end{figure}

\newpage
\twocolumn



