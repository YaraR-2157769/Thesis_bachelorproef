\section{THEORIE}
\subsection{Bayesiaanse statistiek}
In onze opleiding wordt enkel kennisgemaakt met frequentistische statistiek. In deze vorm van statistiek wordt er steeds gekeken naar tellingen. Er wordt ook aangenomen dat er nog niets geweten is over het systeem. Er wordt aan de tellingen begonnen zonder voorgaande kennis. Als je wil weten wat de kans is dat je een vijf gooit met een zeszijdige dobbelsteen, dan gooi je 1000 keer met de dobbelsteen en tel je het aantal keer dat je een vijf gooide.\\ \\
In de Bayesiaanse statistiek wordt de kans geïnterpreteerd als de mate van kennis die vergaard is over het systeem.
In de Bayesiaanse statistiek wordt vaak aangenomen dat er al kennis is van het systeem. Er wordt begonnen met een prior kansverdeling, die modelleert wat er al gekend is van het systeem. Kans wordt gezien als een dynamisch gegeven dat continu verandert op basis van nieuwe beschikbare gegevens. Op basis van deze informatie wordt de kansdichtheidsfunctie continu aangepast. Zo kan er toegewerkt worden naar een posterior kansdichtheid die weergeeft wat er uiteindelijk geweten is over het systeem. Op basis van deze posterior kunnen er probabilistische uitspraken gedaan worden over het systeem. \cite{walker-2005}.\\ \\
In deze statistiek wordt er gewerkt met een belangrijke formule, de stelling van Bayes. Deze is terug te vinden in \cref{for:bayes} \cite{hayes-2024}.
\begin{equation}
    P(H|E)=\frac{P(E|H)\cdot P(H)}{P(E)}
    \label{for:bayes}
\end{equation}
De parameters van \cref{for:bayes} worden overlopen\cite{unknown-author-2023}.
\begin{itemize}
    \item $H$: de hypothese
    \item $E$: het bewijs, de waarneming
    \item $P(H|E)$: de posterior, dit is de kans op de hypothese gegeven het geobserveerde bewijs E. Dit is waar naar opzoek gegaan wordt.
    \item $P(E|H)$: de kans op het observeren van E, gegeven de hypothese. Dit wordt de likelihood genoemd. Dit toont aan hoe compatibel de gemeten data is met de hypothese. 
    \item $P(E)$: dit is de marginale waarschijnlijkheid. Deze is dezelfde voor alle hypotheses
    \item $P(H)$: de prior, dit is de schatting van de kans op een hypothese H, voordat er bewijs waargenomen is.
\end{itemize}
\subsection{Monte Carlo en het metropolis algoritme}
Monte Carlo is een reeks van computationele algoritmes die gebruikt kunnen worden om te samplen uit een kansdichtheid. De term Monte Carlo wijst op het feit dat er gebruik gemaakt wordt van willekeurige gebeurtenissen, dit gaat verder dan het sampelen uit een kansdichtheid, maar dat is voor deze bachelorproef niet relevant. Door veelvuldige herhaling van hetzelfde algoritme, maar met een andere beginpositie kan een kansdichtheid gegenereerd worden \cite{wikipedia-bijdragers-2023}\cite{kenton-2023}. Als er een helder sterrenstelsel staat voor het beeld dat nodig is voor een volledige waarneming van de gravitationele lens, dan kan dat stelsel weggefilterd worden door de parameters van het stelsel, en de parameters van het beeld af te schatten. Als alle parameters gekend zijn, kan het beeld gereconstrueerd worden.\\ \\
In deze bachelorproef wordt er steeds een dataset gesimuleerd. Uit deze dataset worden samples genomen. Die samples worden geïnterpreteerd als de af te schatten parameters. Het nemen van samples wordt gedaan met behulp van Monte Carlo. Omdat er gewerkt wordt met kansverdelingen van meerdere dimensies wordt gebruik gemaakt van Markov-Chain Monte Carlo \cite{unknown-author-2023-MCMC}.\\ \\
Het algoritme dat in deze bachelorproef gebruikt werd, is het Metropolis–Hastings algoritme. Dit is gebaseerd op Markov-Chain Monte Carlo. Het algoritme werkt als volgt\cite{unknown-author-no-date}, \cite{thijssen-2007}:\\ \\
Zij f(x) een functie die evenredig is met de gewenste kansdichtheidsfunctie. Noem f(x) de target functie. Heel vaak komt deze target overeen met de posterior van de stelling van Bayes. Het algoritme gaat als volgt:
\begin{enumerate}
    \item initialisatie: kies een random punt $x_{t}$ als eerste observatie, of kies een punt op basis van eerder verkregen info van het systeem.
    \item herhaal voor elke iteratie t:
    \begin{itemize}
        \item Er wordt een nieuwe kandidaat x' voorgesteld, door bij de vorige waarde een waarde op te tellen die random gekozen wordt uit de normale verdeling met als gemiddelde 0 en als standaardafwijking 0.1 (er zou ook een andere kansverdeling gebruikt kunnen worden, ook de standaardafwijking is afhankelijk van de situatie).
        \item De acceptatieratio $\alpha$, $\alpha=\frac{f(x')}{f(x_{t})}$ wordt bepaald. Dit wordt gebruikt om te bepalen of de nieuwe waarde geaccepteerd wordt, of dat er verder gewerkt wordt met de oorspronkelijke waarde. 
        \item Er wordt bepaald of de waarde geaccepteerd of verworpen wordt. Er wordt een random getal $u$ tussen 0 en 1 gegenereerd. Als $u\leq\alpha$ dan wordt de toestand geaccepteerd. Als $u>\alpha$ dan wordt de kandidaat verworpen en wordt er verder gegaan met de oorspronkelijke waarde.
    \end{itemize}
\end{enumerate}
Naast dit algoritme wordt er ook gewerkt met de emcee library \cite{unknown-author-no-date-emcee}. Deze werkt ook op basis van Markov-Chain Monte Carlo, maar heeft als grote voordeel dat er op voorhand niks geweten moet zijn over de kansverdeling. In de emcee library wordt er gebruik gemaakt van een aantal walkers, die allemaal onafhankelijk van elkaar de kansverdeling gaan afstappen. Zij gaan zelf hun stapgrootte en dergelijke aanpassen aan de situatie. De walkers starten vanop verschillende beginposities en werken toe naar een target kansverdeling. 











