\section*{ABSTRACT}
% Edit below
\textbf{Bij gravitationele lenzing kan het voorkomen dat een object afgebeeld wordt op meerdere beelden. Soms komt het voor dat een beeld achter de lens (een zware massa) staat. Dan kan nuttige informatie over de massaverdeling van de lens niet gevonden worden.\\ \\
Om toch informatie te kunnen bekomen over de zware massa moet het beeld dat achter de zware massa staat gevonden worden. Dat wordt gedaan op drie verschillende manieren, allen gebaseerd op de stelling van Bayes: de kansdichtheid, het Metropolis algoritme en de emcee library. Alle drie de methodes geven gelijkende resultaten. \\ \\
Eerst werd het Metropolis algoritme uitgetest. Dat werkt goed vanaf een voldoende groot aantal samples. Nadien werd met de kansdichtheid en het Metropolis algoritme informatie over een signaal met achtergrondruis verzameld. Ook dit werkte goed. Bij het signaal en de ruis waren er twee onbekende parameters. Dit werd uitgebreid naar zes parameters, de parameters van de zware massa (een sersic). Ook het afschatten van alle parameters van een sersic is gelukt. \\ \\
Het is ook gelukt om de parameters van een systeem bestaande uit twee sersics af te schatten. Er werd een zware massa op een werkelijke achtergrond (gemodelleerd door NASA) geplaatst. Hier werd er opnieuw geprobeerd om de parameters van de zware massa zo goed mogelijk af te schatten, zodat de achtergrond zichtbaar wordt. Dit is niet volledig gelukt, doordat de achtergrond te ruisig was en er op die manier geen nauwkeurige afschatting van de parameters van de sersic gemaakt konden worden. In de toekomst zou voor dit probleem AI gebruikt kunnen worden.}