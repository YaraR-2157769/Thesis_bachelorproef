\section*{ABSTRACT}
% Edit below
\textbf{Bij gravitationele lenzing kan het voorkomen dat een object afgebeeld wordt op meerdere beelden. Bij een reconstructie van het oorspronkelijke object zijn dan alle beelden nodig. Soms komt het voor dat een beeld achter de lens (een zware massa) staat. Dan kan de reconstructie niet gemaakt worden.\\ \\
Om toch een reconstructie te kunnen maken moet er info over het beeld achter de zware massa bekomen worden. Dat wordt gedaan op drie verschillende manieren: de stelling van Bayes, het Metropolis algoritme en de emcee library. Alle drie de methodes geven gelijkende resultaten. \\ \\
Eerst werd het Metropolis algoritme uitgetest. Dat werkt goed vanaf een voldoende groot aantal samples. Nadien werd met de stelling van Bayes en het Metropolis algortime informatie over een signaal met achtergrondruis verzameld. Ook dit werkte goed. Bij het signaal en de ruis waren er twee onbekende paramters. Dit werd uitgebreid naar zes parameters, de parameters van de zware massa (een sersic). Ook het afschatten van de parameters van een sersic is gelukt. \\ \\
Het is ook gelukt om de paramters van een beeld achter een zware massa af te schatten. Op die manier is er info verkregen over het beeld. Er werd geprobeerd om een zware massa op een werkelijke achtergrond (gemodelleerd door NASA) te plaatsen. Hier werd er geprobeerd om de paramters van de zware massa zo goed mogelijk af te schatten, zodat de achtergrond zichtbaar wordt. Dit is niet volledig gelukt, doordat de achtergrond te ruizig was en er op die manier geen nauwkeurige afschatting van de parameters van de sersic gemaakt konden worden.}